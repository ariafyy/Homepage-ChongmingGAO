%%%%%%%%%%%%%%%%%%%%%%%%%%%%%%%%%%%%%%%%%
% "ModernCV" CV and Cover Letter
% LaTeX Template
% Version 1.1 (9/12/12)
%
% This template has been downloaded from:
% http://www.LaTeXTemplates.com
%
% Original author:
% Xavier Danaux (xdanaux@gmail.com)
%
% License:
% CC BY-NC-SA 3.0 (http://creativecommons.org/licenses/by-nc-sa/3.0/)
%
% Important note:
% This template requires the moderncv.cls and .sty files to be in the same 
% directory as this .tex file. These files provide the resume style and themes 
% used for structuring the document.
%
%%%%%%%%%%%%%%%%%%%%%%%%%%%%%%%%%%%%%%%%%

%----------------------------------------------------------------------------------------
%   PACKAGES AND OTHER DOCUMENT CONFIGURATIONS
%----------------------------------------------------------------------------------------

\documentclass[11pt,a4paper,sans]{moderncv} % Font sizes: 10, 11, or 12; paper sizes: a4paper, letterpaper, a5paper, legalpaper, executivepaper or landscape; font families: sans or roman
% \documentclass{ctexart}

 \usepackage{ctex}
\usepackage{xeCJK}
\usepackage{fontspec}

\renewcommand{\baselinestretch}{1.2}

% \usepackage{fontspec}
% \usepackage{xunicode}
% \usepackage{xeCJK}
% \setmainfont{Chancery.ttf}
% \setsansfont{Chancery.ttf}
% \setmonofont{Chancery.ttf}

\setCJKmainfont{FanSong.ttf}
% \setCJKmainfont{SimHei.ttf}
\setCJKsansfont{SimKai.ttf}
% \setCJKmonofont{STSong}
% %\setCJKmathfont{}

\newCJKfontfamily\kaiti{SimKai.ttf}
\newcommand{\kai}[1]{{\kaiti #1}}

\newCJKfontfamily\cuti{SimHei.ttf}
% \newCJKfontfamily\cuti{RuiZiYunZiKuRuiSongCuGB-2.ttf}
\newcommand{\cu}[1]{{\cuti #1}}

% \usepackage[scale=0.85, top=15mm, bottom=13mm]{geometry}
\usepackage[scale=0.85]{geometry}

\moderncvstyle{classic} % CV theme - options include: 'casual' (default), 'classic', 'oldstyle' and 'banking'
\moderncvcolor{blue} % CV color - options include: 'blue' (default), 'orange', 'green', 'red', 'purple', 'grey' and 'black'
\usepackage{lipsum} % Used for inserting dummy 'Lorem ipsum' text into the template
\usepackage{ulem}

% \usepackage[scale=0.85]{geometry} % Reduce document margins
%\setlength{\hintscolumnwidth}{3cm} % Uncomment to change the width of the dates column
%\setlength{\makecvtitlenamewidth}{10cm} % For the 'classic' style, uncomment to adjust the width of the space allocated to your name

%----------------------------------------------------------------------------------------
%   NAME AND CONTACT INFORMATION SECTION
%----------------------------------------------------------------------------------------

\firstname{\kai{高崇铭} (Chongming} % Your first name
\familyname{Gao)
% \\\vspace{3mm}
% \LARGE{HomePage: ~ https://chongminggao.me}
} % Your last name

% All information in this block is optional, comment out any lines you don't need
%\title{Curriculum Vitae}
% \address{2 rue Fran\c cois Verny}{Chengdu, China 29806}
% \address{UESTC, Chengdu, China}
% \mobile{(+33) 7 86 33 40 99}
%\phone{(000) 111 1112}
%\fax{(000) 111 1113}

% \usepackage{fontawesome}
\email{chongming.gao@gmail.com}
\mobile{(+86) 18064850580}
\extrainfo{WeChat: 619082231}

\homepage{https://chongminggao.me}%{staff.org.edu/$f$jsmith}% The first argument is %the url for the clickable link, the second argument is the url displayed in the %template - this allows special characters to be displayed such as the tilde in this %example

%\photo[70pt][0.4pt]{picture} % The first bracket is the picture height, the second is %the thickness of the frame around the picture (0pt for no frame)
% \quote{Seeking a PhD Placement}

%----------------------------------------------------------------------------------------
% \moderncvhead{1}

\newfontfamily\chancery[Ligatures=TeX]{Chancery.ttf}
\renewcommand*\namefont{\chancery\fontsize{30}{48}\selectfont}

% \renewcommand*\namefont{\fontfamily{pzc}\fontsize{40}{48}\selectfont}
% \renewcommand*\titlefont{\fontfamily{pzc}\fontsize{20}{24}\selectfont}
% \renewcommand*\addressfont{\fontfamily{pzc}\selectfont}
% \renewcommand*\sectionfont{\fontfamily{pzc}\fontsize{20}{24}\selectfont}
% \renewcommand*{\addressfont}{\fontfamily{ppl}\fontsize{14}{18}\selectfont}

% \renewcommand*{\addressstyle}[1]{{\addressfont\textcolor{color2}{#1}}}

% \definecolor{blue}{rgb}{0.11764706,  0.68235294,  0.85882353} % skyblue
\definecolor{blue}{rgb}{0.22,0.45,0.70} % skyblue
% \definecolor{blue}{rgb}{0.19607843,  0.45098039,  0.8627451}
\newcommand{\blue}[1]{{\textcolor{blue}{{} #1}}}

\definecolor{gray}{rgb}{0.2,0.2,0.2} % skyblue
\newcommand{\gray}[1]{{\textcolor{gray}{{} #1}}}

% \definecolor{red}{rgb}{1,0,0}
% \newcommand{\red}[1]{{\textcolor{red}{{}#1}}}

\definecolor{red}{HTML}{f74545} % red
\newcommand{\red}[1]{{\textcolor{red}{{} #1}}}

\begin{document}
\makecvtitle % Print the CV title

%----------------------------------------------------------------------------------------
%   EDUCATION SECTION
%----------------------------------------------------------------------------------------

\vspace{-7mm}
\section{\cu{研究领域}}

\cventry{}{数据挖掘、搜索推荐,与人工智能}{}{}{主要致力于应用诸如图分析、因果推断的方法来对时空数据挖掘、对话推荐系统等场景进行探索}{}


\section{\cu{实习经历}}

% \cventry{After 2020\\Ph.D.}{\cu{\sout{昆士兰大学, 信息科技与电子工程学院}}}{}{}{}{
% \sout{导师}:\href{https://sites.google.com/site/dbhongzhi/}{\blue{\sout{阴红志}}.}\vspace{1mm}\\
% \sout{研究领域: 推荐系统、NLP}\vspace{1mm}\\
% \red{由于疫情影响,经费减少,学校缩招,取消读博打算。}
% }
% \vspace{1mm}

\cventry{2020.07--至今\\科研实习}{\cu{快手}}{}{}{}{实习导师:\kai{\href{http://staff.ustc.edu.cn/~hexn/}{\blue{\kai{何向南}}, 江鹏, 李彪}}.\vspace{1mm}\\
实习方向: 交互式推荐以及对话推荐系统在快手场景中的研究。}
\vspace{1mm}

% \cventry{2019.03--2019.09\\科研实习}{\cu{阿里巴巴} AI Labs}{}{}{}{实习导师:\kai{\href{https://scholar.google.com/citations?user=N4rQYDAAAAAJ&hl=en}{\blue{王浩}},\href{http://www.zaiqing.net/}{\blue{聂再清}}}.\vspace{1mm}\\
% 实习成果: 解决了天猫精灵的一个推荐问题。}
% \vspace{1mm}

\cventry{2019.03--2019.09\\科研实习}{\cu{阿里巴巴} AI Labs}{}{}{}{实习导师:\kai{王浩, \href{https://sites.google.com/view/hongzhi-yin/home}{\blue{\kai{阴红志}}}, \href{http://air.tsinghua.edu.cn/EN/team-detail.html?id=35&classid=8}{\blue{聂再清}}}.\vspace{1mm}\\
实习成果: 探究解决了天猫精灵的多人共享下的推荐问题。}
\vspace{1mm}

% \cventry{2016.09--2019.03}{University of Electronic Science and Technology of China}{}{}{}
% {M.Eng. in Department of Computer Science and Technology. Supervisor:\href{http://dm.uestc.edu.cn}{\blue{Junming Shao}}.\\}

\section{\cu{教育背景}}

\cventry{2020.09--至今\\博士}{\cu{中国科学技术大学,信息科学技术学院}}{}{}{}
% {计算机科学与工程学院 \vspace{1mm}\\
{
科研导师: \href{http://staff.ustc.edu.cn/~hexn/}{\blue{\kai{何向南}}}。\vspace{1mm}\\
研究方向: 对话推荐系统,因果推荐。}
\vspace{1mm}

\cventry{2016.09--2019.06\\硕士}{\cu{电子科技大学,计算机科学与工程学院}}{}{}{}
% {计算机科学与工程学院 \vspace{1mm}\\
{
科研导师: \href{http://dm.uestc.edu.cn}{\blue{\kai{邵俊明}}}。\vspace{1mm}\\
毕设论文: 《轨迹语义表征与地点推荐研究》,校级优秀毕设。}
\vspace{1mm}

% \cventry{2016.09--2019.03}{University of Electronic Science and Technology of China}{}{}{}
% {M.Eng. in Department of Computer Science and Technology.\vspace{1mm}\\
% Supervisor:\href{http://dm.uestc.edu.cn}{\blue{Junming Shao}}.\\}

\cventry{2012.09--2016.06\\本科}{\cu{电子科技大学,英才实验学院(电子科技大学实验项目学院)}}{}{}{}
{GPA: 3.81, 排名: 9/72 \vspace{1mm}\\
科研导师:\href{http://dm.uestc.edu.cn}{\blue{\kai{邵俊明}}}, 大三学年开始科研。\vspace{1mm}\\
毕设论文: 《基于双同步聚类的双聚类算法及其在基因表达数据上的应用》,校级优秀毕设。}  % Arguments not required can be left empty
\vspace{1mm}
% \cvitemwithcomment{Arabic}{Native Speaker}{asdf}

%----------------------------------------------------------------------------------------
%   WORK EXPERIENCE SECTION
%----------------------------------------------------------------------------------------

\section{\cu{顶会Tutorial}}
\cventry{\red{\fbox{\textbf{Tutorial}}}\\(3 hours)}{\small RecSys 2021 Tutorial on Conversational Recommendation: Formulation, Methods, and Evaluation}{}{}{}{\small \cu{雷文强, \underline{高崇铭}},  Maarten de Rijke, \vspace{1mm}\\
\emph{RecSys 2021 Tutorial}.\vspace{1mm}\\
在推荐系统领域的国际顶会RecSys 2021上进行了一场3个小时的讲习班,为对话推荐系统的基本概念、最新进展、以及评测方法做了一个全面的剖析。} 
\vspace{1mm}

\section{\cu{论文发表}}

% \cvitem{Accepted}{\small Junming Shao, Chongming Gao, Wei Zeng, Jingkuan Song, Qinli Yang, Synchronization-inspired Co-clustering and Its Application to Gene Expression Data.}

% \cventry{ICDM'20\\在投中...}{*\small A paper about identifying users behind shared accounts}{}{}{}{\small \kai{\underline{高崇铭}, 王浩, 余俊良, 曹涌, 聂再清, 阴红志}\vspace{1mm}\\
% Submitted to IEEE International Conference on Data Mining (\textbf{ICDM'20})\vspace{1mm}\\
% (会议等级: CCF B), 在投中。\vspace{1mm}\\
% 内容简介: 天猫精灵作为一款语音助手,其推荐系统不考虑多用户共享使用同一设备的场景。本工作首先通过点播内容将使用同一设备的多个用户区分开,再针对识别出用户进行针对性推荐。其识别效果与推荐效果都超过SOTA算法。
% }




\cventry{AI Open\\\red{\fbox{\textbf{综述}}}}{\small Advances and Challenges in Conversational Recommender Systems: A Survey}{}{}{}{\small \cu{\underline{高崇铭}, 雷文强, 何向南},  Maarten de Rijke, Tat-Seng Chua\vspace{1mm}\\
\emph{AI Open. Vol. 2. (2021) 100-126}.\vspace{1mm}\\
工作简介: 一篇对话推荐系统的综述,总结了对话推荐中的5个挑战,针对每个挑战列出了现有进展和成果。并提出了5个未来可探索的方向。} 
\vspace{1mm}

\cventry{IS'20}{\small Semantic Trajectory Representation and Retrieval via Hierarchical Embedding}{}{}{}{\small \cu{\underline{高崇铭}, 张众, 黄晨, 杨勤丽, 邵俊明}\vspace{1mm}\\
Information Sciences (\textbf{IS'20})。 \vspace{1mm}\\
(中科院SCI期刊分区2020年 大类: 1区, 小类: 1区, CiteScore: 6.90, 影响因子: 5.524)。\vspace{1mm}\\
工作简介: 轨迹数据通常不定长,使得表征与挖掘都困难。本文用一种动态聚类的方法将轨迹表征成为一个层次语义网络。在此层次语义网络上,用网络Embedding的方式来重新表征轨迹,使得区域与轨迹之间的语义相似度被重新定义。基于此方式的轨迹检索效果好于基于传统DTW、LCSS、EDR等方法。}

\cventry{EMNLP'20}{\small Revisiting Representation Degeneration Problem in Language Modeling}{}{}{}{\small \cu{张众, \underline{高崇铭}, 许璁, 苗蕊, 杨勤丽, 邵俊明}\vspace{1mm}\\
\emph{Findings of EMNLP, 2020.}\vspace{1mm}\\
(会议等级:CORE2020 Rank: A, CCF B). \vspace{1mm}\\
工作简介: 提出了一个拉普拉斯约束项来解决NLP中语言模型的表征退化问题。
}
\vspace{1mm}

\cventry{DASFAA'19\red{\fbox{\cu{最佳论文!}}}}
{\small Towards Robust Arbitrarily Oriented Subspace Clustering}{}{}{}{\small \cu{张众, \underline{高崇铭}, 刘崇志, 杨勤丽, 邵俊明}\vspace{1mm}\\
 International Conference on Database Systems for Advanced Applications (\textbf{DASFAA'19}), \vspace{1mm}\\
(会议等级: CCF B)。\vspace{1mm}\\
工作简介: 传统子空间聚类的方法总是受到局部和全局噪声的干扰,且运行效率低。本文提供一种全新的子空间搜索方法思路,效果鲁班,速度快。
}
\vspace{1mm}

\cventry{DASFAA'19\\(Short Paper)}{\small \mbox{BLOMA: Explain Collaborative Filtering via Boosted Local Rank-One Matrix Approximation}}{}{}{}{\small \cu{\underline{高崇铭}, 袁帅, 张众, 阴红志, 邵俊明}\vspace{1mm}\\
 International Conference on Database Systems for Advanced Applications (\textbf{DASFAA'19}), \vspace{1mm}\\
 (会议等级: CCF B)\vspace{1mm}\\
工作简介: 基于矩阵分解的推荐系统存在一个大问题——分解出来的隐向量没有意义,不具有解释性。本工作提出一种全新的可解释性推荐算法,能对一次推荐自动作出解释:“本次推荐满足了您对于中餐(60\%)以及海鲜(\%30)的喜好。” 方法原理:利用秩一分解,每次从用户-商品矩阵中采样出最大的“尚未解释”的分量,利用side information将其解释并消去。
}
\vspace{1mm}

\cventry{KBS'19}{\small Semantic Trajectory Compression via Multi-resolution Synchronization-based Clustering}{}{}{}{\small \cu{\underline{高崇铭}, 赵奕, 吴睿智, 杨勤丽, 邵俊明}\vspace{1mm}\\
Knowledge-Based Systems (\textbf{KBS'19}), \vspace{1mm}\\
(中科院SCI期刊分区2020年 大类: 1区, 小类: 1区, CiteScore: 7.01, 影响因子: 5.101)。\vspace{1mm}\\
工作简介: 轨迹数据量大、不定长、采样率不一致等特性使得其存储和表示成为难题。本工作利用一种合适的动态性距离算法,巧妙地将轨迹全局地表示成一个多层次网络,从而达到灵活表征、快速传输存取的目的。
}
\vspace{1mm}


\cventry{ICDM'19}{\small Online Budgeted Least Squares with Unlabeled Data}{}{}{}{\small \cu{黄晨, 李培炎, \underline{高崇铭}, 杨勤丽, 邵俊明}\vspace{1mm}\\
IEEE International Conference on Data Mining (\textbf{ICDM'19}), \vspace{1mm}\\
(会议等级: CCF B).\vspace{1mm}\\
工作简介: 在线的半监督聚类通常要求动态维护拉普拉斯矩阵,这样复杂度非常高。本文提出一种在线Budgeted式的最小二乘法,使得在线半监督聚类变得很高效。本文提供了理论证明,这种在线方式的误差比起离线算法是有界的。
}
\vspace{1mm}

\cventry{ICDM'19}{\small Generating Reliable Friends via Adversarial Training to Improve Social Recommendation}{}{}{}{\small \cu{余俊良, 高旻, 阴红志, 李俊东, \underline{高崇铭}, 王覃泳}\vspace{1mm}\\
IEEE International Conference on Data Mining (\textbf{ICDM'19}), \vspace{1mm}\\
(会议等级: CCF B).\vspace{1mm}\\
工作简介: 在目前考虑社交网络的推荐系统中,由于社交网络的极度稀疏,很多理论上奏效的方法效果并不理想。本文用对抗生成的方式,为每一个用户生成一些靠谱的朋友,再基于这些靠谱的朋友继续用对抗生成的方式做出推荐。整个过程动态循环。该方式超出了所有social recommendation的baseline方法。
}
\vspace{1mm}


\cventry{DASFAA'19\\(Short Paper)}{\small SemiSync: Semi-supervised Clustering by Synchronization}{}{}{}{\small \cu{张众, 康迪迪, \underline{高崇铭}, 邵俊明}\vspace{1mm}\\
 International Conference on Database Systems for Advanced Applications (\textbf{DASFAA'19}), \vspace{1mm}\\
(会议等级 CCF B)\vspace{1mm}\\
工作简介: 在半监督数据中有一种特殊形式: 已知必须连(Must-link)的边与必不能连(Cannot-link)的边,需要聚类得到最终簇。本文提出一种基于动态聚类的全新半监督聚类算法。利用“吸引或者排斥”的交互法则,让所有点与周围邻居动态交互,得到稳态即为聚类结果。
}
\vspace{1mm}


\cventry{ICDM'17}{\small Synchronization-inspired Co-clustering and Its Application to Gene Expression Data}{}{}{}{\small \cu{邵俊明, \underline{高崇铭}, 曾伟, 宋井宽, 杨勤丽}\vspace{1mm}\\
IEEE International Conference on Data Mining (\textbf{ICDM'17}), \vspace{1mm}\\
(会议等级: CCF B).\vspace{1mm}\\
工作简介: 在例如“用户-商品”(或“基因-蛋白质”)表示的矩阵数据上,传统聚类算法只能对用户或者商品进行聚类。本文提出一种“双聚类”的算法,同时对用户和商品进行聚类,得到有意义且解释性强的子簇(子矩阵)。
}
\vspace{1mm}








% \vspace{1mm}



\section{\cu{获奖情况}}

\cvitem{2019}{\small DASFAA'19 (CCF B){最佳论文奖}。} 
\cvitem{2019}{\small 电子科技大学{校级优秀硕士毕业论文} (86/3744)。} 
\cvitem{2019}{\small 电子科技大学{校级优秀硕士毕业生}。} 
\cvitem{2016}{\small 电子科技大学校级优秀毕设答辩专场\textbf{荣获最高分}(94分)。}
\cvitem{2016}{\small 电子科技大学{校级优秀本科毕业生}。英才实验学院第(10/72)名。}
\cvitem{2014}{\small 美国数学建模大赛一等奖(M奖)。}
\cvitem{2013}{\small 国家数学建模大赛四川省一等奖。}
\cvitem{2012}{\small 唐立新奖学金,60/25000.}
\cvitem{2012}{\small 电子科技大学在云南省录取最高分(614分)。}
%----------------------------------------------------------------------------------------
%   COMPUTER SKILLS SECTION
%----------------------------------------------------------------------------------------

\section{\cu{擅长语言}}

% \cvitem{Basic}{VHDL, UML}
\cvitem{编程}{Python, \textsc{Matlab}, \textsc{Java}, C/C++, \LaTeX, HTML5+CSS3+Javascript}
\cvitem{排版}{Adobe Illustrator, Adobe Photoshop}


%----------------------------------------------------------------------------------------
%   LANGUAGES SECTION
%----------------------------------------------------------------------------------------

% \section{\cu{Languages}}
% \begin{small}
% \cvitemwithcomment{Arabic}{Native Speaker}{}
% \cvitemwithcomment{French}{Near Native}{excellent command}
% \cvitemwithcomment{English}{Near Native}{excellent command}
% \cvitemwithcomment{Chinese}{Intermediate}{good working knowledge}
% \cvitemwithcomment{Japanese}{Intermediate}{good working knowledge}
% \end{small}


%----------------------------------------------------------------------------------------
%   INTERESTS SECTION
%----------------------------------------------------------------------------------------
%\bigskip

\section{\cu{兴趣}}

\renewcommand{\listitemsymbol}{-~} % Changes the symbol used for lists

\cvitem{}{读书, 羽毛球, 跑步, 游泳, 拍照.}
%\cvlistdoubleitem{Robotics}{}


%----------------------------------------------------------------------------------------
%   COVER LETTER
%----------------------------------------------------------------------------------------

% To remove the cover letter, comment out this entire block

%\clearpage

%\recipient{HR Departmnet}{Corporation\\123 Pleasant Lane\\12345 City, State} % Letter recipient
%\date{\today} % Letter date
%\opening{Dear Sir or Madam,} % Opening greeting
%\closing{Sincerely yours,} % Closing phrase
%\enclosure[Attached]{curriculum vit\ae{}} % List of enclosed documents

%\makelettertitle % Print letter title

%\lipsum[1-3] % Dummy text

%\makeletterclosing % Print letter signature

%----------------------------------------------------------------------------------------
\end{document}